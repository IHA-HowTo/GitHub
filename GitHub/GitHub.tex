\documentclass{article}

\title{HowTo -- GitHub}
\author{Rasmus Bækgaard, 10893}
\date{\today}


\begin{document}

\maketitle

\section{Setup a GitHub account}

\begin{itemize}
\item Go to \url{github.com}

\item Create a user.
\subitem Make something you can use for work aswell

\item Select a free plan. 
It will stay free and you can even get a private (hidden) repository for free (more on that later)

\item Choose the first option \textbf{Set up Git}

	\begin{itemize}
	\item[] Follow the instructions on your screen, but listed here as well:

	\item Download and install the application for GitHub at \url{https://windows.github.com/} 
	\subitem Mac users can use \url{https://mac.github.com/}
	\subitem Linux users -- you know what to do.

	\item Enter mail / username and password when promted

	\item Enter public email and the name that should appear (please use your real name)

	\item If you encounter a 3th step, skip this.
	\end{itemize}

\end{itemize}

Now -- that wasn't so bad?


\section{Your first project}

\begin{itemize}
\item Click the '+' to create a new project

\item Give it a name and select a filter

	\begin{itemize}
	\item Filters can, and should be, expanded to fit each project/repository.
	\item The filter prevents unwanted files to be shared with others
	\item[] Example: You compile a \texttt{LaTeX} and creates a \texttt{.pdf}-file.
	You don't want to merge that with existing files since you have no clue of how a \texttt{.pdf}-file is written and the \texttt{LaTeX}-tool will do it for you.
	\end{itemize}

\item Notice two files have been created:
	
	\begin{itemize}
	\item \texttt{.gitattributes}
	\item \texttt{.gitignore}

	\item[] These files some merge strategies (which we shall not dwell on) and which files (or entire folders) should be ignored.
	\end{itemize}

\item Go ahead and make a Visual Studio project or just a simple \texttt{.txt}-file, save it and return to the GitHub application.

\item Notice the new bar called "Uncommitted changes" and click on "Show".

\item To the right you will see the files saved which are not on the \texttt{.gitignore} list.

\item You can expand the files to see what has been added (marked with a small '+'), removed (marked with a small '-') and unchanged (not marked).

\item Commit files with a summary (and if you want to, a description) -- this is a must!
\subitem If you don't want to commit 

\item Hit the "check mark" to commit the files.
\subitem This is only a commit. It's a "safe this as ready to upload" and will also serve as a bookmark which you can revert back to.

\item Publish the content by clicking in the top right corner on "Publish Repository" and give it a description.

\item If you add another file or modify an existing one, a "Uncomitted changes" will show up.
\subitem Commit it and hit "Sync" to push the commit to GitHub.com





\end{itemize}

\end{document}